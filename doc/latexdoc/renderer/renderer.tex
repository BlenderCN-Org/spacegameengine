\documentclass{article}
\usepackage{listings}

\begin{document}
\lstset{language=C++}

\section{$<$sge/renderer/renderer.hpp$>$}

sge::renderer is a thin wrapper around the backend's functionality.
It is provided to abstract different backends like opengl and direct3d.
Renderer provides mechanisms for
\begin{itemize}
\item{Creating other abstract interfaces like textures, buffers and shaders}
\item{Customizing renderer settings}
\item{Retrieving some values specific to the renderer like capabilities}
\item{Rendering geometry}
\end{itemize}

\begin{lstlisting}
void begin_rendering();
\end{lstlisting}
Effects: Begins a rendering pass.
This must be called before any rendering can be done.
It also invokes the clearing mechanism.

\begin{lstlisting}
void end_rendering();
\end{lstlisting}
Effects: Ends a rendering pass.
This must be called to present the scene to the chosen destination.

\begin{lstlisting}
void render(
	vertex_buffer_ptr vb,
 	index_buffer_ptr ib,
 	vertex_buffer::size_type first_vertex,
 	vertex_buffer::size_type num_vertices,
 	indexed_primitive_type::type ptype,
 	index_buffer::size_type primitive_count,
 	index_buffer::size_type first_index):
\end{lstlisting}
Parameters:
\begin{itemize}
\item vertex\_buffer\_ptr vb: Pointer to a vertex buffer providing the geometry
\item index\_buffer\_ptr ib: Pointer to an index buffer providing the indices for this call
\item vertex\_buffer::size\_type first\_vertex: The index of the first vertex to use
\item vertex\_buffer::size\_type num\_vertices: Number of vertices to use starting from first vertex
\item indexed\_primitive\_type::type ptype: Indexed primitive type to render
\item index\_buffer::size\_type primitive\_type: Number of primitves to draw
\item index\_buffer::size\_type first\_index: Starting index to use in the index buffer
\end{itemize}
Effects: Renders a number of indexed primitives.
Note: first\_vertex and num\_vertices are provided for compatibility to D3D9 and should be set to 0 and vb-$>$size() respectively if the caller is unsure.

\begin{lstlisting}
void render(
	vertex_buffer_ptr vb,
	vertex_buffer::size_type first_vertex,
	vertex_buffer::size_type num_vertices,
 	nonindexed_primitive_type::type ptype);
\end{lstlisting}
Parameters:
\begin{itemize}
\item vertex\_buffer\_ptr vb: Pointer to a vertex buffer providing the geometry
\item vertex\_buffer::size\_type first\_vertex: The index of the first vertex to use
\item vertex\_buffer::size\_type num\_vertices: The numer of vertices to use starting from first vertex
\item nonindexed\_primitive\_type::type ptype: Nonindexed primitive type to render
\end{itemize}
Effects: Renders a number of nonindexed primitives.
Note: The number of primitives is directly inferred from num\_vertices.

\begin{lstlisting}
void set_int_state(
	int_state::type state,
	int_type value);
\end{lstlisting}
Parameters:
\begin{itemize}
\item int\_state::type state: The int state to set
  \begin{itemize}
  \item int\_state::stencil\_clear\_val: Sets the value the stencil buffer will be cleared with
  \end{itemize}
\item int\_type value: The int value to set
\end{itemize}
Effects: Sets the int value denoted by state to value.

\begin{lstlisting}
void set_float_state(
	float_state::type state,
	float_type value);
\end{lstlisting}
Parameters:
\begin{itemize}
\item float\_state::type state: The float state to set
  \begin{itemize}
  \item float\_state::zbuffer\_clear\_val: Sets the value the z buffer will be cleared with
  \item fog\_start: Sets the fog starting range
  \item fog\_end: Sets the fog ending range
  \item fog\_density: Sets the fog's density
  \end{itemize}
\item float\_type value: The float value to set
\end{itemize}
Effects: Sets the float value denoted by state to value.

\begin{lstlisting}
void set_bool_state(
	bool_state::type state,
	bool_type value);
\end{lstlisting}
Parameters:
\begin{itemize}
\item bool\_state::type state: The bool state to set
  \begin{itemize}
  \item bool\_state::clear\_zbuffer: Sets if the z buffer should be cleared
  \item bool\_state::clear\_backbuffer: Sets if the back buffer should be cleared
  \item bool\_state::clear\_stencil: Sets if the stencil buffer should be cleared
  \item bool\_state::enable\_fog: Sets if fog should be enabled
  \item bool\_state::enable\_stencil: Sets if the stencil buffer should be enabled
  \item bool\_state::enable\_alpha\_blending: Sets if alpha blending should be enabled
  \item bool\_state::enable\_zbuffer: Sets if the z buffer should be enabled
  \item bool\_state::enable\_lighting: Sets if lighting should be enabled
  \item bool\_state::enable\_culling: Sets if culling should be enabled
  \end{itemize}
\item bool\_type value: The bool value to set
\end{itemize}
Effects: Enables or disables specifc options.

\begin{lstlisting}
void set_color_state(
	color_state::type state,
	color value);
\end{lstlisting}
Parameters:
\begin{itemize}
\item color\_state::type state: The color state to set
  \begin{itemize}
  \item color\_state::clear\_color: The back buffer's clear color
  \item color\_state::ambient\_light\_color: The ambient light's color
  \item color\_state::fog\_color: The fog's color
  \end{itemize}
\item color value: The color value to set
\end{itemize}
Effects: Sets a specific color denoted by state to value.
	
\begin{lstlisting}
void set_cull_mode(
	cull_mode::type value);
\end{lstlisting}
Parameters:
\begin{itemize}
\item cull\_mode::type value: The cull mode to set
  \begin{itemize}
  \item cull\_mode::back: Cull the back faces
  \item cull\_mode::front: Cull the front faces
  \end{itemize}
\end{itemize}
Effects: Sets the cull mode to value

\begin{lstlisting}
void set_depth_func(
	depth_func::type value);
\end{lstlisting}
Parameters:
\begin{itemize}
\item depth\_func::type value: The z buffer comparison function.
  Note: depth\_func is a namespace alias to compare\_func.
  \begin{itemize}
  \item depth\_func::never: Never let pixels pass
  \item depth\_func::less: Only let pixels pass whose z value is less than the current pixel's z value
  \item depth\_func::equal: Only let pixels pass whose z value is equal to the current pixel's z value
  \item depth\_func::less\_equal: Only let pixels pass whose z value is less or equal to the current pixel's z value
  \item depth\_func::greater: Only let pixels pass whose z value is greater than the current pixel's value
  \item depth\_func::not\_equal: Only let pixels pass whose z value is not equal to the current pixel's value
  \item depth\_func::greater\_equal: Only let pixels pass whose z value is greater or equal to the current pixel's value
  \item depth\_func::always: Always let pixels pass
  \end{itemize}
\end{itemize}
Effects: Sets the z buffer comparsion function to value.

\begin{lstlisting}
void set_stencil_func(
	stencil_func::type value);
\end{lstlisting}
Parameters:
\begin{itemize}
\item stencil\_func::type value: The stencil buffer comparsion function.
  Note: stencil\_func is a namespace alias to compare\_func. See set\_depth\_func for possible values.
\end{itemize}
Effects: Sets the stencil buffer comparsion function to value.

\begin{lstlisting}
void set_fog_mode(
	fog_mode::type value);
\end{lstlisting}
Parameters:
\begin{itemize}
\item fog\_mode::type value: The fog mode.
  \begin{itemize}
  \item fog\_mode::linear: The fog increases linearly
  \item fog\_mode::exp: The fog increases exponentially
  \item fog\_mode::exp2: The fog increases exponentially square
  \end{itemize}
\end{itemize}
Effects: Sets the fog mode to value.

\begin{lstlisting}
void set_blend_func(
	source_blend_func::type src,
	dest_blend_func::type dest);
\end{lstlisting}
\begin{itemize}
\item source\_blend\_func::type src: The source blending function.
  \begin{itemize}
  \item source\_blend\_func::zero: Use zero
  \item source\_blend\_func::one: Use one
  \item source\_blend\_func::dest\_color: Use the destination color
  \item source\_blend\_func::inv\_dest\_color: Use 1 - the destination color
  \item source\_blend\_func::src\_alpha: Use the source's alpha value
  \item source\_blend\_func::inv\_src\_alpha: Use 1 - the source's alpha value
  \item source\_blend\_func::dest\_alpha: Use the destination's alpha value
  \item source\_blend\_func::inv\_dest\_alpha: Use 1 - the destination's alpha value
  \item source\_blend\_func::src\_alpha\_sat: TODO
  \end{itemize}
\item dest\_blend\_func::type dest: The destination blending function.
  \begin{itemize}
  \item dest\_blend\_func::zero: Use zero
  \item dest\_blend\_func::one: Use one
  \item dest\_blend\_func::src\_color: Use the source color
  \item dest\_blend\_func::inv\_src\_color: Use 1 - the source color
  \item dest\_blend\_func::src\_alpha: Use the source's alpha value
  \item dest\_blend\_func::inv\_src\_alpha: Use 1 - the source's alpha value
  \item dest\_blend\_func::dest\_alpha: Use the destination's alpha value
  \item dest\_blend\_func::inv\_dest\_alpha: Use 1 - the destination's alpha value
  \end{itemize}
\end{itemize}
Effects: Sets the source blend function to src and the destination blend function to dest.

\begin{lstlisting}
void set_draw_mode(
	draw_mode::type value);
\end{lstlisting}
Parameters:
\begin{itemize}
\item draw\_mode::type value: The draw mode.
  \begin{itemize}
  \item draw\_mode::point: Draw only the primitives' points.
  \item draw\_mode::line: Draw only the primitives' lines (same as wireframe mode).
  \item draw\_mode::fill: Fill primitives (default).
  \end{itemize}
\end{itemize}
Effects: Sets the draw mode to value.

\begin{lstlisting}
void set_texture(
	texture_base_ptr tex,
	stage_type stage = 0);
\end{lstlisting}
Parameters:
\begin{itemize}
\item texture\_base\_ptr tex: A pointer to a texture\_base object. May be a normal texture, a cube texture, a volume texture or a render target.
The value \emph{no\_texture} can be used to disable this texture level.
\item stage\_type stage: The texture stage to set.
\end{itemize}
Effects: Sets the texture stage denoted by stage to the texture denoted by tex or deactivates the texture stage if tex is \emph{no\_texture}.

\begin{lstlisting}
void set_material(
	const material& mat);
\end{lstlisting}
Parameters:
\begin{itemize}
\item material mat: The material to use.
\end{itemize}
Effects: Sets the current material to mat.

\begin{lstlisting}
void transform(
	const math::space_matrix& mat);
\end{lstlisting}
Parameters:
\begin{itemize}
\item space\_matrix mat: The matrix to use for transformation.
\end{itemize}
Effects: Sets the matrix to use for the world transformation step to mat.

\begin{lstlisting}
void projection(
	const math::space_matrix& mat);
\end{lstlisting}
Parameters:
\begin{itemize}
\item space\_matrix mat: The matrix to use for projection.
\end{itemize}
Effects: Sets the matrix to ues for the projection transformation step to mat.

\begin{lstlisting}
void set_render_target(
	texture_ptr target);
\end{lstlisting}
Parameters:
\begin{itemize}
\item texture\_ptr target: The render target to use. May be default\_render\_target.
\end{itemize}
Effects: Sets the current render target to use to target. If default\_render\_target is used the default render target is restored.

\begin{lstlisting}
void set_viewport(
	const viewport& v);
\end{lstlisting}
Parameters:
\begin{itemize}
\item viewport v: The viewport to use.
\end{itemize}
Effects: Sets the current viewport to v.

\begin{lstlisting}
void enable_light(
	light_index index,
	bool enable);
\end{lstlisting}
Parameters:
\begin{itemize}
\item light\_index index: The light with index 'index' to enable or disable.
\item bool enable: True enables the light and false disables it.
\end{itemize}
Effects: Enables or disables the light denoted by light\_index.

\begin{lstlisting}	
void set_light(
	light_index index,
	const light& l);
\end{lstlisting}
Parameters:
\begin{itemize}
\item light\_index index: The light index to set.
\item light l: The light to use.
\end{itemize}
Effects: Sets the light denoted by index to l.
Note: Setting a light does not enable or disable it. Disabling a light does not clear a light level.

\begin{lstlisting}
void set_texture_stage_op(
	stage_type stage,
	texture_stage_op::type op,
	texture_stage_op_value::type value);
\end{lstlisting}
Parameters:
\begin{itemize}
\item stage\_type stage: The texture stage to set.
\item texture\_stage\_op::type op: The operation to set.
  \begin{itemize}
  \item texture\_stage\_op::color: Set the color operation
  \item texture\_state\_op::alpha: Set the alpha opreation
  \end{itemize}
\item texture\_stage\_op\_value::type value: The value to use.
  \begin{itemize}
  \item texture\_stage\_op\_value::arg0: Replace the result by arg0
  \item texture\_stage\_op\_value::modulate: Multiply arg0 and arg1
  \item texture\_stage\_op\_value::modulate2x: Multiply arg0 and arg1 and multiply that by 2
  \item texture\_stage\_op\_value::modulate4x: Multiply arg0 and arg1 and multiply that by 4
  \item texture\_stage\_op\_value::add: Add arg0 and arg1
  \item texture\_stage\_op\_value::add2x: TODO
  \item texture\_stage\_op\_value::add4x: TODO
  \item texture\_stage\_op\_value::substract: Substract arg0 and arg1
  \item texture\_stage\_op\_value::add\_signed: TODO
  \item texture\_stage\_op\_value::add\_signed2x: TODO
  \item texture\_stage\_op\_value::interpolate: TODO
  \end{itemize}
\end{itemize}
Effects: Sets the texture stage operation denoted by stage and op to value.
Note: This function is superseeded by shaders.

\begin{lstlisting}
void set_texture_stage_arg(
	stage_type stage,
	texture_stage_arg::type arg,
	texture_stage_arg_value::type value);
\end{lstlisting}
Parameters:
\begin{itemize}
\item stage\_type stage: The texture stage to set.
\item texture\_stage\_arg::type arg: The argument to set.
  \begin{itemize}
  \item texture\_stage\_arg::rgb0: Set rgb0 argument
  \item texture\_stage\_arg::rgb1: Set rgb1 argument
  \item texture\_stage\_arg::rgb2: Set rgb2 argument
  \item texture\_stage\_arg::alpha0: Set alpha0 argument
  \item texture\_stage\_arg::alpha1: Set alpha1 argument
  \item texture\_stage\_arg::alpha2: Set alpha2 argument
  \end{itemize}
\item texture\_stage\_arg\_value::type value: The value to use.
  \begin{itemize}
  \item texture\_stage\_arg\_value::current: Use the calculated value from the previous stage
  \item texture\_stage\_arg\_value::texture: Use the texture's value from the current stage
  \item texture\_stage\_arg\_value::constant: Use a set constant from arg2 (TODO)
  \end{itemize}
\end{itemize}
Effects: Sets the texture stage argument denoted by stage and arg to value.
Note: This function is superseeded by shaders.

\begin{lstlisting}
void push();
\end{lstlisting}
Effects: Pushes the current renderer settings except textures, shaders and buffers to the stack.

\begin{lstlisting}
void pop();
\end{lstlisting}
Effects: Pops the current renderer settings except textures, shaders and buffers from the stack.

\begin{lstlisting}
glsl::program_ptr create_glsl_program(
	const std::string& vertex_shader_source
		= no_shader,
	const std::string& pixel_shader_source
		= no_shader);
\end{lstlisting}
Paremeters:
\begin{itemize}
\item std::string vertex\_shader\_source: Use this string as the glsl vertex shader's source code.
no\_shader can be used to create a shader that uses no vertex shader.
\item std::string pixel\_shader\_source: Use this string as the glsl pixel shader's source code.
no\_shader can be used to create a shader that uses no pixel shader.
\end{itemize}
Effects: Creates a glsl program that uses a pair of a vertex and a pixel shader.
Either one of them can be specified with no\_shader in which case the shader is disabled.

\begin{lstlisting}
void set_glsl_shader(glsl::program_ptr prog);
\end{lstlisting}
Parameters:
\begin{itemize}
\item glsl::program\_ptr prog: The program to set. Can be no\_program to set disable shaders.
\end{itemize}
Returns: The created glsl program.
Effects: Sets the current glsl program to use to prog. If no\_program is specified shaders will be disabled.

\begin{lstlisting}
render_target_ptr get_render_target() const;
\end{lstlisting}
Returns: The current active render target.

\begin{lstlisting}
texture_ptr create_texture(
	texture::const_pointer data,
	const texture::dim_type& dim,
	const filter_args& filter,
	resource_flag_t flags
		= resource_flags::default_);
\end{lstlisting}
Parameters:
\begin{itemize}
\item texture::const\_pointer data: The data to fill the texture with.
\item texture::dim\_type dim: The texture's dimension.
\item filter\_args filter: The filter the texture uses.
\item resource\_flag\_t flags: The resource flags.
\end{itemize}
Returns: The created texture.
Effects: Creates a new texture with the specified parameters.

\begin{lstlisting}
volume_texture_ptr create_volume_texture(
	volume_texture::const_pointer data,
	const volume_texture::box_type& box,
	const filter_args& filter,
	resource_flag_t flags
		= resource_flags::default_);
\end{lstlisting}
Paramters:
\begin{itemize}
\item volume\_texture::const\_pointer data: The data to fill the texture with.
TODO
\end{itemize}
\end{document}
